
\documentclass[10pt]{article} 
\usepackage{geometry}                		
\geometry{a4paper,margin=2cm}                   		
\usepackage{graphicx}	
\usepackage{xcolor}				
\usepackage{amssymb}
\usepackage{amsmath}
\usepackage{mathtools}
\usepackage[utf8]{inputenc}
\usepackage{amsthm}
\usepackage{bm}
\usepackage{xcolor}
\newcommand{\comm}[1]{{\color{red}{#1}}}
\newcommand{\commb}[1]{{\color{blue}{#1}}}
\usepackage{amsthm}
\newtheorem{prop}{Proposition}
\newtheorem{thm}{Theorem}
\newtheorem{prf}{Proof}
\newtheorem{exl}{Example}

\usepackage{algorithmic}
\usepackage{algorithm}
\usepackage{cite}
\usepackage{booktabs}
\usepackage{hyperref}
\makeatletter
\newcommand{\doubletilde}[1]{{%
  \mathpalette\double@tilde{#1}%
}}
\newcommand{\double@tilde}[2]{%
  \sbox\z@{$\m@th#1\tilde{#2}$}%
  \ht\z@=.9\ht\z@
  \tilde{\box\z@}%
}
\makeatother
\usepackage{amsthm}
\newtheorem{rmrk}{Remark}
\newtheorem{thrm}{Theorem}
\newtheorem{lmma}{Lemma}
\begin{document}
\title{Exact formulae and matrix-less eigensolvers for \\ preconditioned block banded symmetric Toeplitz matrices}
\author{Sven-Erik Ekstr\"om         \and
        Isabella Furci \and
        Stefano Serra-Capizzano 
}\date{\today}
\maketitle
\begin{abstract}
In a series of papers...
\end{abstract}
\section{Introduction}
\label{sec:intro}
A matrix $\textbf{A}_n$  of the form
\begin{equation*}
\textbf{A}_n=\left[A_{i-j}\right]_{i,j=1}^{n}=\begin{bmatrix}
A_0 & A_{-1} & A_{-2} & \cdots & \cdots & A_{-(n-1)} \\
A_1 & \ddots & \ddots & \ddots & & \vdots\\
A_2 & \ddots & \ddots & \ddots & \ddots & \vdots\\
\vdots & \ddots & \ddots & \ddots & \ddots & A_{-2}\\
\vdots & & \ddots & \ddots & \ddots & A_{-1}\\
A_{n-1} & \cdots & \cdots & A_2 & A_1 & A_0
\end{bmatrix},
\end{equation*}
where $A_{-(n-1)},\ldots,A_{n-1}$ are blocks in $\mathbb{C}^{s\times s}$, is said to be %an $s\times s$ block Toeplitz matrix or simply 
a block Toeplitz matrix. Note that the size of $\textbf{A}_n$ is $N(n,s)=sn$. 
%with  $A_k$ being matrices belonging to $\mathbb{C}^{s\times s}$, $k \in \mathbb{Z}$. 

We say that $\phi:[-\pi,\pi]\to\mathbb C^{s\times s}$ is a complex matrix-valued Lebesgue integrable function if all its components $\phi_{i,j}:[-\pi,\pi]\to \mathbb{C}$, $i,j=1,\dots, s$, are complex-valued Lebesgue integrable functions.
The $n$th block Toeplitz matrix generated by $\phi$ is defined as
\begin{equation*}
T_n(\phi)=\bigl[\hat \phi_{i-j}\bigr]_{i,j=1}^n,
\end{equation*}
where the quantities $\hat \phi_k \in \mathbb{C}^{s\times s}$ are the Fourier coefficients of $\phi$,  that is, 
\begin{equation}\label{fourier-coeff}
\hat \phi_{k}=\frac1{2\pi}\int_{-\pi}^{\pi}\phi(\theta)\,{\rm e}^{-\mathbf{i}k\theta}{\rm d}\theta,\qquad k\in\mathbb Z.
\end{equation}
We refer to $\{T_n(\phi)\}_n$ as the block Toeplitz sequence generated by $\phi$, which in turn is called the generating function or the symbol of $\{T_n(\phi)\}_n$.
\section{Main Results}
\section{Numerical Experiments}
\subsection{Standard Precondition}
 different examples
\subsection{$\mathbb{Q}_p$ FEM}
In~\cite{ekstrom181} closed formulae for the eigenvalues of $\mathbb{Q}_p$ Lagrange finite element stiffness and mass matrices, $K_n^{p}$ and $M_n^{p}$, were given. 


Now consider the problem
\begin{align}
-\Delta u=\lambda u
\end{align}
with Dirichlet boudnary conditions. In discretized form, using $\mathbb{Q}_p$ FEM, is
\begin{align}
K_n^{p}\mathbf{u}_n=\lambda M_n^{p}\mathbf{u}_n.
\end{align}
Thus solving the generalized eigenvalue problem we have
\begin{align}
L_n^{p}\mathbf{u}_n=\lambda \mathbf{u}_n.
\end{align}
where
\begin{align}
L_n^{p}=(M_n^{p})^{-1}K_n^{p},
\end{align}
We here give closed formulae for the eigenvalues of $L_n^{p}$.

\comm{insert correct scalings}

\comm{insert algorithm, see Code/Experiments/newblock.m}

In the following algorithm the symbol $\boldsymbol{\kappa}$ represents respectively the symbols $\mathbf{f},\mathbf{g},\mathbf{e},\mathbf{r}$ for the different matrices.
\begin{enumerate}
\item Sample symbol $\boldsymbol{\kappa}(\theta)$ with the grid $\tau_{n-1}^{0,\pi}$, that is,
\begin{align}
\theta_{j,n+1}=\frac{(j-1)\pi}{n},\quad j=1,\ldots, n+1.
\end{align}
For each sampling a matrix of size $p\times p$ is returned and a local eigenvalue problem is solved, giving $p$ eigenvalues. These are sorted in non-decreasing order and stored in $p$ different vectors $\boldsymbol{\mu}^{(i)}$. This results in $(n+1)p$ samplings, wheras the matrix only has $np$ eigenvalues, so $p$ samplings has to be discared.
\item For eigenvalues belonging to the eigenvalue function $\lambda^{(1)}(\mathbf{\boldsymbol{\kappa}})$ (the smallest eigenvalues from each sampling) choose the eigenvalues $\boldsymbol{\mu}_j^{(1)}$, $j=2,\ldots, n$. This corresponds to using the grid $\tau_{n-1}$.
\item[3.1.] For symbol $\mathbf{f}$ 
\begin{itemize}
	\item For all $p$, and $1<q\leq p$
\begin{itemize}
\item $q$ even: Eigenvalues belonging to $\lambda^{(q)}(\mathbf{f})$ choose $\boldsymbol{\mu}_j^{(q)}$, $j=2,\ldots, n+1$.
\item $q$ odd: Eigenvalues belonging to $\lambda^{(q)}(\mathbf{f})$ choose $\boldsymbol{\mu}_j^{(q)}$, $j=1,\ldots, n$.
\end{itemize}
\end{itemize}
\item[3.2.] For symbol $\mathbf{g}$
\begin{itemize}
\item For $p$ even, and $1<q\leq p$ 
\begin{itemize}
\item $q$ even: Eigenvalues belonging to $\lambda^{(q)}(\mathbf{f})$ choose $\boldsymbol{\mu}_j^{(q)}$, $j=2,\ldots, n+1$.
\item $q$ odd: Eigenvalues belonging to $\lambda^{(q)}(\mathbf{f})$ choose $\boldsymbol{\mu}_j^{(q)}$, $j=1,\ldots, n$.
\end{itemize}
\item For $p$ odd, and $1<q\leq p$


\begin{itemize}
	\item Define
\begin{align}
\hat{p}=\begin{cases}
p,&(p+1)/2 \text{ odd,}\\
p-2,&(p+1)/2 \text{ even.}
\end{cases}\nonumber
\end{align}
\item $q$ even: Eigenvalues belonging to $\lambda^{(q)}(\mathbf{g})$ choose
\begin{align}
\boldsymbol{\mu}_j^{(q)},\quad j=
\begin{cases}
1,\ldots, n,&1<q\leq (\hat{p}+1)/2,\\
2,\ldots, n+1,&(\hat{p}+1)/2<q\leq p.
\end{cases}\nonumber
\end{align}
\item $q$ odd: Eigenvalues belonging to $\lambda^{(q)}(\mathbf{g})$ choose
\begin{align}
\boldsymbol{\mu}_j^{(q)},\quad j=
\begin{cases}
2,\ldots, n+1,&1<q\leq (\hat{p}+1)/2,\\
1,\ldots, n,&(\hat{p}+1)/2<q\leq p.
\end{cases}\nonumber
\end{align}
% \item $q\leq (\hat{p}+1)/2$ even:  Eigenvalues belonging to $\lambda^{(q)}(\mathbf{g})$ choose $\boldsymbol{\mu}_j^{(q)}$, $j=1,\ldots, n$.
% \item $q\leq (\hat{p}+1)/2$ odd:  Eigenvalues belonging to $\lambda^{(q)}(\mathbf{g})$ choose $\boldsymbol{\mu}_j^{(q)}$, $j=2,\ldots, n+1$.
% \item $(\hat{p}+1)/2<q\leq p$ even: Eigenvalues belonging to $\lambda^{(q)}(\mathbf{g})$ choose $\boldsymbol{\mu}_j^{(q)}$, $j=2,\ldots, n+1$.
% \item $(\hat{p}+1)/2<q\leq p$ odd: Eigenvalues belonging to $\lambda^{(q)}(\mathbf{g})$ choose $\boldsymbol{\mu}_j^{(q)}$, $j=1,\ldots, n$.
\end{itemize}
\end{itemize}
\item[3.3.] For symbol $\mathbf{e}$

\item[3.4.] For symbol $\mathbf{r}=\mathbf{g}^{-1}\mathbf{f}$
\begin{itemize}
\item For $p$ even, and $1\leq q\leq p$
\begin{itemize}
\item Add $\boldsymbol{\mu}_{n+1}^{(1)}$ to the eigenvalues beloning to $\lambda^{(1)}(\mathbf{r})$ chosen in Step 2.
\item $q$ even: Eigenvalues belonging to $\lambda^{(q)}(\mathbf{r})$ choose $\boldsymbol{\mu}_j^{(q)}$, $j=2,\ldots, n$.
\item $q$ odd and $q>1$: Eigenvalues belonging to $\lambda^{(q)}(\mathbf{r})$ choose $\boldsymbol{\mu}_j^{(q)}$, $j=1,\ldots, n+1$.
\end{itemize}
\item For $p$ odd, and $1<q\leq p$
\begin{itemize}
\item $q$ even: Eigenvalues belonging to $\lambda^{(q)}(\mathbf{r})$ choose $\boldsymbol{\mu}_j^{(q)}$, $j=1,\ldots, n+1$.
\item $q$ odd: Eigenvalues belonging to $\lambda^{(q)}(\mathbf{r})$ choose $\boldsymbol{\mu}_j^{(q)}$, $j=2,\ldots, n$.
\end{itemize}
\end{itemize}
	% b=mod(p,2);
	% aeL{1}=aeLs(1,2:n+1-b)';
	% if p>1
	% 	for qq=2:2:p
	% 		aeL{qq}=aeLs(qq,2-b:n+b)';
	% 	end
	% 	for qq=3:2:p
	% 		aeL{qq}=aeLs(qq,1+b:n+1-b)';
	% 	end
	% end
\end{enumerate}
\begin{align}
\begin{array}{rrrrrrrrrrrr}
K_n^{p}&M_n^{p}&H_n^{p}&L_n^{p}\\
\end{array}\nonumber
\end{align}
\section{Conclusions}
\bibliography{References}{}
\bibliographystyle{plain}

\end{document}