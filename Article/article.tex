
\documentclass[10pt]{article} 
\usepackage{geometry}                		
\geometry{a4paper,margin=2cm}                   		
\usepackage{graphicx}	
\usepackage{xcolor}				
\usepackage{amssymb}
\usepackage{amsmath}
\usepackage{mathtools}
\usepackage[utf8]{inputenc}
\usepackage{amsthm}
\usepackage{bm}
\usepackage{xcolor}
\newcommand{\comm}[1]{{\color{red}{#1}}}
\newcommand{\commb}[1]{{\color{blue}{#1}}}
\usepackage{amsthm}
\newtheorem{prop}{Proposition}
\newtheorem{thm}{Theorem}
\newtheorem{prf}{Proof}
\newtheorem{exl}{Example}

\usepackage{algorithmic}
\usepackage{algorithm}
\usepackage{cite}
\usepackage{booktabs}
\usepackage{hyperref}
\makeatletter
\newcommand{\doubletilde}[1]{{%
  \mathpalette\double@tilde{#1}%
}}
\newcommand{\double@tilde}[2]{%
  \sbox\z@{$\m@th#1\tilde{#2}$}%
  \ht\z@=.9\ht\z@
  \tilde{\box\z@}%
}
\makeatother
\usepackage{amsthm}
\newtheorem{rmrk}{Remark}
\newtheorem{thrm}{Theorem}
\newtheorem{lmma}{Lemma}
\begin{document}
\title{Exact formulae and matrix-less eigensolvers for \\ preconditioned block banded symmetric Toeplitz matrices}
\author{Sven-Erik Ekstr\"om         \and
        Isabella Furci \and
        Stefano Serra-Capizzano 
}\date{\today}
\maketitle
\begin{abstract}
In a series of papers...
\end{abstract}
\section{Introduction}
\label{sec:intro}
A matrix $\textbf{A}_n$  of the form
\begin{equation*}
\textbf{A}_n=\left[A_{i-j}\right]_{i,j=1}^{n}=\begin{bmatrix}
A_0 & A_{-1} & A_{-2} & \cdots & \cdots & A_{-(n-1)} \\
A_1 & \ddots & \ddots & \ddots & & \vdots\\
A_2 & \ddots & \ddots & \ddots & \ddots & \vdots\\
\vdots & \ddots & \ddots & \ddots & \ddots & A_{-2}\\
\vdots & & \ddots & \ddots & \ddots & A_{-1}\\
A_{n-1} & \cdots & \cdots & A_2 & A_1 & A_0
\end{bmatrix},
\end{equation*}
where $A_{-(n-1)},\ldots,A_{n-1}$ are blocks in $\mathbb{C}^{s\times s}$, is said to be %an $s\times s$ block Toeplitz matrix or simply 
a block Toeplitz matrix. Note that the size of $\textbf{A}_n$ is $N(n,s)=sn$. 
%with  $A_k$ being matrices belonging to $\mathbb{C}^{s\times s}$, $k \in \mathbb{Z}$. 

We say that $\phi:[-\pi,\pi]\to\mathbb C^{s\times s}$ is a complex matrix-valued Lebesgue integrable function if all its components $\phi_{i,j}:[-\pi,\pi]\to \mathbb{C}$, $i,j=1,\dots, s$, are complex-valued Lebesgue integrable functions.
The $n$th block Toeplitz matrix generated by $\phi$ is defined as
\begin{equation*}
T_n(\phi)=\bigl[\hat \phi_{i-j}\bigr]_{i,j=1}^n,
\end{equation*}
where the quantities $\hat \phi_k \in \mathbb{C}^{s\times s}$ are the Fourier coefficients of $\phi$,  that is, 
\begin{equation}\label{fourier-coeff}
\hat \phi_{k}=\frac1{2\pi}\int_{-\pi}^{\pi}\phi(\theta)\,{\rm e}^{-\mathbf{i}k\theta}{\rm d}\theta,\qquad k\in\mathbb Z.
\end{equation}
We refer to $\{T_n(\phi)\}_n$ as the block Toeplitz sequence generated by $\phi$, which in turn is called the generating function or the symbol of $\{T_n(\phi)\}_n$.
\section{Main Results}
\section{Numerical Experiments}
\subsection{Standard Precondition}
 different examples
\subsection{$\mathbb{Q}_p$ FEM}
In~\cite{ekstrom181} closed formulae for the eigenvalues of $\mathbb{Q}_p$ Lagrange finite element stiffness and mass matrices, $K_n^{p}$ and $M_n^{p}$, were given. 


Now consider the problem
\begin{align}
-\Delta u=\lambda u
\end{align}
with Dirichlet boudnary conditions. In discretized form, using $\mathbb{Q}_p$ FEM, is
\begin{align}
K_n^{p}\mathbf{u}_n=\lambda M_n^{p}\mathbf{u}_n.
\end{align}
Thus solving the generalized eigenvalue problem we have
\begin{align}
L_n^{p}\mathbf{u}_n=\lambda \mathbf{u}_n.
\end{align}
where
\begin{align}
L_n^{p}=(M_n^{p})^{-1}K_n^{p},
\end{align}
We here give closed formulae for the eigenvalues of $L_n^{p}$.

\comm{insert correct scalings}

\comm{insert algorithm, see Code/Experiments/newblock.m}
\section{Conclusions}
\bibliography{References}{}
\bibliographystyle{plain}

\end{document}